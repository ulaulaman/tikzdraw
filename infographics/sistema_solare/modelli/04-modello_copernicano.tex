\documentclass{standalone}
%
\usepackage{tikz}
\usetikzlibrary{backgrounds,shapes.callouts}
\usepackage{tkz-euclide}
\usepackage{xcolor}
%\usepackage{pgfplots}
%\usepackage{freetikz}
%\usetikzlibrary{shapes}
%\definecolor{eduinafred}{HTML}{BB2D58}
%\definecolor{eduinafblu}{HTML}{1D71B8}
%
\definecolor{space}{HTML}{0A2543}
\definecolor{mercury}{HTML}{846549}
\definecolor{venus}{HTML}{BB9765}
\definecolor{earth}{HTML}{0089FA}
\definecolor{mars}{HTML}{DC7B4E}
\definecolor{jupiter}{HTML}{A79476}
\definecolor{saturn}{HTML}{DBBD9B}
\definecolor{saturnring}{HTML}{857C73}
\definecolor{uranus}{HTML}{b1d8dd}
\definecolor{neptune}{HTML}{799bc1}
\definecolor{pluto}{HTML}{ceaa8a}
\definecolor{dida}{HTML}{FFDE00}
\definecolor{title}{HTML}{FBA706}
%
%\definecolor{glasses}{HTML}{08663D}
\definecolor{moon}{HTML}{AFAFAF}
%\definecolor{craterm}{HTML}{616060}
%\definecolor{linem}{HTML}{DBDBDB}
%\definecolor{core3}{HTML}{FFD016}
%
\usepackage{fontspec}
\setmainfont{Open Dyslexic}
%\setmainfont{Montserrat Medium}
%
\title{Modello copernicano}
\begin{document}
	\tikzset{
		partial ellipse/.style args = {#1:#2:#3}{insert path={+ (#1:#3) arc (#1:#2:#3)}},
		notice/.style  = { draw, ellipse callout, callout relative pointer={#1} },
	}
	\begin{tikzpicture}[background rectangle/.style={fill=white},show background rectangle,>={[inset=0,angle'=27]Stealth}]
		\def\rsun{1}
		\def\rmer{0.2}
		\def\rven{0.5}
		\def\ret{0.53}
		\def\rmn{0.2}
		\def\rms{0.28}
		\def\rj{1}
		\def\rsat{1}
		\def\rur{1}
		\def\rnp{1}
		\def\rp{0.1}
		%
		\def\d{2}
		%
		\draw [use as bounding box] (-20,20) -| (20,20) |- (20,-68) -| (-20,-68);
		%title
		\begin{scope}
			\draw [black,ultra thick,fill=title] (-18,18.5) rectangle (18,11.5);
			\node (example-textwidth-2) [align=center, text width=40cm, color=black, font=\fontsize{85pt}{86pt}\selectfont] at (0,15) {Il modello copernicano};
		\end{scope}
		%
		\begin{scope}[shift={(-10,7)}]
			\node at (23,0) {\includegraphics[width=5cm]{img/carl_sagan}};
			\node (example-textwidth-2) [notice={(3,0.5)}, ultra thick, right, align=center, text width=12cm, color=black, fill=white, font=\fontsize{23pt}{24pt}\selectfont] at (1,-1) {Grazie all'impegno del suo allievo \textbf{Georg Joachim Rethicus}, il modello di \textbf{Nicolaus Copernicus}, o \textbf{Copernico}, venne pubblicato poco dopo la sua morte, nel 1543.};
		\end{scope}
		%
		\begin{scope}[shift={(-3,-25)}]
			\draw [fill=space] (-14,15) rectangle (20,-15);
			\tkzDefPoint(0,0){E} %Earth
			\tkzDefShiftPoint[E](0.5*\d,0){L0} %Moon
			\tkzDefShiftPoint[E](2*\d,0){S0} %Sun
			\tkzDefShiftPoint[S0](0.5*\d,0){Mr} %Mercury
			\tkzDefShiftPoint[S0](0.7*\d,0){V0} %Venus
			\tkzDefShiftPoint[S0](4.5*\d,0){M0} %Mars
			\tkzDefShiftPoint[S0](5.3*\d,0){J0} %Jupiter
			\tkzDefShiftPoint[S0](6*\d,0){R0} %Saturn
			%
			\tkzDrawCircle[color=moon,ultra thick](E,L0)
			\tkzDrawCircle[color=earth,ultra thick](S0,E)
			\tkzDrawCircle[color=mercury,ultra thick](S0,Mr)
			\tkzDrawCircle[color=venus,ultra thick](S0,V0)
			\tkzDrawCircle[color=mars,ultra thick](S0,M0)
			\tkzDrawCircle[color=jupiter,ultra thick](S0,J0)
			\tkzDrawCircle[color=saturn,ultra thick](S0,R0)
			%
			\tkzDefShiftPoint[E](0:\ret){E1}
			\tkzDrawCircle[color=black,ultra thick,fill=earth](E,E1)
			\tkzDefPointBy[rotation= center E angle -90](L0) \tkzGetPoint{l0}
			\node [above, xshift=2.2cm, text width=10cm, color=white, font=\fontsize{15pt}{16pt}\selectfont] at (E1) {Terra};
			%
			\tkzDefShiftPoint[l0](0:\rmer){L1}
			\tkzDrawCircle[color=black,ultra thick,fill=moon](l0,L1)
			\node [above, xshift=3cm, yshift=-0.5cm, text width=10cm, color=white, font=\fontsize{15pt}{16pt}\selectfont] at (L1) {Luna};
			%
			\tkzDefShiftPoint[S0](0:\ret){S1}
			\tkzDrawCircle[color=black,ultra thick,fill=white](S0,S1)
			\node [above, xshift=2.5cm, text width=10cm, color=white, font=\fontsize{15pt}{16pt}\selectfont] at (S1) {Sole};
			%
			\tkzDefShiftPoint[Mr](0:\rmer){mr}
			\tkzDrawCircle[color=black,ultra thick,fill=mercury](Mr,mr)
			\node [above, xshift=5.5cm, text width=10cm, color=white, font=\fontsize{15pt}{16pt}\selectfont] at (mr) {Mercurio};
			%
			\tkzDefPointBy[rotation= center S0 angle 110](V0) \tkzGetPoint{v0}
			\tkzDefShiftPoint[v0](0:\rmer){V1}
			\node [above, xshift=2.5cm, text width=10cm, color=white, font=\fontsize{15pt}{16pt}\selectfont] at (V1) {Venere};
			\tkzDrawCircle[color=black,ultra thick,fill=venus](v0,V1)
			%
			\tkzDefPointBy[rotation= center S0 angle 35](M0) \tkzGetPoint{m0}
			\tkzDefShiftPoint[m0](0:\rmer){M1}
			\tkzDrawCircle[color=black,ultra thick,fill=mars](m0,M1)
			\node [above, xshift=2.5cm, text width=10cm, color=white, font=\fontsize{15pt}{16pt}\selectfont] at (M1) {Marte};
			%
			\tkzDefShiftPoint[J0](0:\rmer){J1}
			\tkzDrawCircle[color=black,ultra thick,fill=jupiter](J0,J1)
			\node [above, xshift=2.5cm, text width=10cm, color=white, font=\fontsize{15pt}{16pt}\selectfont] at (J1) {Giove};
			%
			\tkzDefPointBy[rotation= center S0 angle -40](R0) \tkzGetPoint{r0}
			\tkzDefShiftPoint[r0](0:\rmer){R1}
			\tkzDrawCircle[color=black,ultra thick,fill=saturn](r0,R1)
			\node [above, xshift=2.5cm, text width=10cm, color=white, font=\fontsize{15pt}{16pt}\selectfont] at (R1) {Saturno};
			\draw [color=black,ultra thick] (-14,15) rectangle (20,-15);
		\end{scope}
		%
		\begin{scope}[shift={(-10,-5)}]
			\draw [fill=dida, ultra thick] (6.7,2) rectangle (28,-2);
			\node at (0,-2) {\includegraphics[width=10cm]{img/de_revolutionibus_1543}};
			\node (example-textwidth-2) [right, align=left, text width=21cm, color=black, font=\fontsize{23pt}{24pt}\selectfont] at (7,0) {Era basato sul modello di \textbf{Aristarco da Samo} probabilmente giunto a conoscenza di Copernico grazie a studi islamici e calcoli astronomici.};
		\end{scope}
		%
		\begin{scope}[shift={(-10,-43)}]
			\node at (23,0) {\includegraphics[width=6cm]{img/copernico}};
			\node (example-textwidth-2) [notice={(2,0.5)}, ultra thick, right, align=center, text width=14cm, color=black, fill=white, font=\fontsize{23pt}{24pt}\selectfont] at (0.5,-1) {Lo schema non rispetta esattamente le distanze, ma potete confrontarlo con il modello di \textbf{Tycho Brahe} per apprezzarne la profonda differenza filosofica!};
		\end{scope}
		%
		\begin{scope}[shift={(-10,-50)}]
			\node at (0,-2) {\includegraphics[width=9cm]{img/osiander}};
			\draw [fill=dida, ultra thick] (6.7,1.5) rectangle (28.2,-3.5);
			\node (example-textwidth-2) [right, align=left, text width=21cm, color=black, font=\fontsize{23pt}{24pt}\selectfont] at (7,-1) {Cio' che salvo' l'opera fu l'aggiunta di \textbf{Andreas Osiander} di una prefazione non autorizzata alla prima edizione in cui si sosteneva che quello era un modello matematico senza alcuna pretesa di voler rappresentare la realta'.};
		\end{scope}
		%
		\begin{scope}[shift={(-10,-58)}]
			\node at (23,0) {\includegraphics[width=5cm]{img/carl_sagan}};
			\node (example-textwidth-2) [notice={(2.3,0.2)}, ultra thick, right, align=center, text width=15cm, color=black, fill=white, font=\fontsize{23pt}{24pt}\selectfont] at (-1.5,-1) {La prima dimostrazione largamente accettata del modello copernicano arrivo' nel 1729 quando \textbf{James Bradley} annuncio' la scoperta dell'aberrazione della luce, un fenomeno legato proprio al moto della Terra.};
		\end{scope}
		%
		\begin{scope}[shift={(-10,-66)}]
			\node at (27,0) () {\includegraphics[width=3.7cm]{img/licenza}};
			\node at (18,-0.1) {\textcolor{black}{\fontsize{14}{15}\selectfont Testo e illustrazioni: @ulaulaman - Gianluigi Filippelli}};
		\end{scope}
	\end{tikzpicture}
%
\end{document}