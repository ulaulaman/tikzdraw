\documentclass{standalone}
%
\usepackage{tikz}
%
\usepackage{tkz-euclide}
\usetkzobj{all}
%
\usepackage{dsfont}
%
\usepackage{xcolor}
%
\definecolor{space}{HTML}{0A2543}
\definecolor{earth}{HTML}{0089FA}
\definecolor{dida}{HTML}{FFDE00}
\definecolor{title}{HTML}{FBA706}
%
\usepackage{amsmath}
%
\usepackage{fontspec}
\setmainfont{Open Dyslexic}
%
\title{Un fatto curioso sui semicerchi}
\begin{document}
	\begin{tikzpicture}
		\draw [use as bounding box,color=white] (-0.2,15.2) -| (30.2,15.2) |- (30.2,-60) -| (-0.2,-60);
		%title
		\draw [black,ultra thick,fill=title] (0,7) rectangle (30,15);
		\node (example-textwidth-2) [right, align=center, text width=30cm, color=black, font=\fontsize{90pt}{91pt}\selectfont] at (0,11) {Un fatto curioso sui semicerchi};
		\def\a{5}
		\def\b{9}
		\begin{scope}[shift={(0,4)}]
			\tkzDefPoint(15,0){C1}
			\tkzDefShiftPoint[C1](\a,0){A}
			\tkzDefShiftPoint[C1](-\a,0){A1}
			\tkzDefShiftPoint[C1](0,-\a-\b){C2}
			\tkzDefShiftPoint[C2](\b,0){B}
			\tkzDefShiftPoint[C2](-\b,0){B1}
			%
			\tkzDrawSector[rotate,ultra thick,fill=earth](C1,A)(-180)
			\tkzDrawSector[rotate,ultra thick,fill=earth](C2,B)(180)
			%
			\tkzCircumCenter(A,A1,B)
			\tkzGetPoint{O}
			\tkzDrawCircle(O,A1)
			%
			\tkzDefLine[perpendicular=through A1,K=-1](A1,C1)
			\tkzGetPoint{x}
			\tkzInterLL(A1,x)(C2,B)
			\tkzGetPoint{H}
			\tkzInterLC(A1,H)(O,A1)
			\tkzGetPoints{A2}{r}
			\tkzDrawSegments(A1,A2)
			
			\tkzMarkRightAngle(C1,A1,A2)
			\tkzDrawPoints(C1,C2)
			\tkzLabelPoints[font=\fontsize{18pt}{19pt}](H,B)
			\tkzLabelPoints[above,font=\fontsize{18pt}{19pt}](A1)
			\tkzLabelPoints[below,font=\fontsize{18pt}{19pt}](B1,A2)
			%
			\tkzLabelSegment[font=\fontsize{18pt}{19pt}\selectfont](C1,A1){\emph{a}}
			\tkzLabelSegment[font=\fontsize{18pt}{19pt}\selectfont](C1,A){\emph{a}}
			\tkzLabelSegment[font=\fontsize{18pt}{19pt}\selectfont](C2,B){\emph{b}}
			\tkzLabelSegment[font=\fontsize{18pt}{19pt}\selectfont](C2,H){\emph{a}}
			\tkzLabelSegment[right,font=\fontsize{18pt}{19pt}\selectfont](A1,H){\emph{a + b}}
			\tkzLabelSegment[font=\fontsize{18pt}{19pt}\selectfont](H,B1){\emph{b - a}}
			\tkzLabelSegment[left,font=\fontsize{18pt}{19pt}\selectfont](H,A2){\emph{x}}
		\end{scope}
		%
		\begin{scope}[shift={(0,-17)}]
			\draw [fill=space, thick] (3.5,0) rectangle (23,-6);
			\draw [fill=dida, thick] (1.8,0.5) rectangle (11.2,-0.5);
			\node [right, align=left, color=black, font=\fontsize{23pt}{24pt}\selectfont] at (2,0) {Teorema delle corde};
			\node [right, align=left, color=white, font=\fontsize{23pt}{24pt}\selectfont] at (4,-2) {$A_1H : HB = B_1H : HA2$};
			\node [right, align=left, color=white, font=\fontsize{23pt}{24pt}\selectfont] at (4,-4) {$A_1H = a+b = HB \Rightarrow b-a = B_1H = HA_2 = x$};
		\end{scope}
		%
		\begin{scope}[shift={(0,-26)}]
			\tkzDefPoint(15,0){C1}
			\tkzDefShiftPoint[C1](\a,0){A}
			\tkzDefShiftPoint[C1](-\a,0){A1}
			\tkzDefShiftPoint[C1](0,-\a-\b){C2}
			\tkzDefShiftPoint[C2](\b,0){B}
			\tkzDefShiftPoint[C2](-\b,0){B1}
			%
			\tkzDrawSector[rotate,ultra thick,fill=earth](C1,A)(-180)
			\tkzDrawSector[rotate,ultra thick,fill=earth](C2,B)(180)
			%
			\tkzCircumCenter(A,A1,B)
			\tkzGetPoint{O}
			\tkzDrawCircle(O,A1)
			%
			\tkzDefLine[perpendicular=through A1,K=-1](A1,C1)
			\tkzGetPoint{x}
			\tkzInterLL(A1,x)(C2,B)
			\tkzGetPoint{H}
			\tkzInterLC(A1,H)(O,A1)
			\tkzGetPoints{A2}{r}
			\tkzDrawSegments(A1,A2 A2,A)
			
			\tkzMarkRightAngle(C1,A1,A2)
			\tkzLabelPoints[above,font=\fontsize{18pt}{19pt}](A1,A)
			\tkzLabelPoints[below,font=\fontsize{18pt}{19pt}](A2)
			%
			\tkzLabelSegment[font=\fontsize{18pt}{19pt}\selectfont](A1,A){2\emph{a}}
			\tkzLabelSegment[right,font=\fontsize{18pt}{19pt}\selectfont](A1,A2){2\emph{b}}
			\tkzLabelSegment[right,font=\fontsize{18pt}{19pt}\selectfont](A2,A){2\emph{r}}
			\tkzDrawPoint(O)
		\end{scope}
		%
		\begin{scope}[shift={(0,-47)}]
			\draw [fill=space, thick] (3.5,0) rectangle (23,-10);
			\draw [fill=dida, thick] (1.8,0.6) rectangle (11.2,-0.5);
			\node [right, align=left, color=black, font=\fontsize{23pt}{24pt}\selectfont] at (2,0) {Teorema di Pitagora};
			\node [right, align=left, color=white, font=\fontsize{23pt}{24pt}\selectfont] at (4,-2) {$(2a)^2 + (2b)^2 = (2r)^2 \Rightarrow a^2 + b^2 = r^2$};
			\node [right, align=left, color=white, font=\fontsize{23pt}{24pt}\selectfont] at (4,-4) {$A_a = \frac{1}{2} \pi a^2, \qquad A_b = \frac{1}{2} \pi b^2$};
			\node [right, align=left, color=white, font=\fontsize{23pt}{24pt}\selectfont] at (4,-6) {$A_c = \frac{1}{2} \pi r^2$};
			\node [right, align=left, color=white, font=\fontsize{23pt}{24pt}\selectfont] at (4,-8) {$A_a + A_b = \frac{1}{2} \pi (a^2 + b^2) = \frac{1}{2} \pi r^2 = \frac{1}{2} A_c, \, \forall a, b \in \mathds{R}$};
		\end{scope}
		%
		\begin{scope}[shift={(0,-59)}]
			\node at (27,0) () {\includegraphics[width=3.7cm]{licenza}};
			\node at (18,-0.1) {\textcolor{black}{\fontsize{14}{15}\selectfont Testo e illustrazioni: @ulaulaman - Gianluigi Filippelli}};
		\end{scope}
	\end{tikzpicture}
%
\end{document}