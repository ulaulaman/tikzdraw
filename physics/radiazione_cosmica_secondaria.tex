\documentclass{standalone}
%
\usepackage{tikz}
\usetikzlibrary{backgrounds}
\usetikzlibrary{calc}
\usetikzlibrary{decorations.pathmorphing}
\usetikzlibrary{bending,arrows.meta}
\usetikzlibrary{shapes.callouts}
\usetikzlibrary{shapes.geometric}
\usetikzlibrary{shapes.arrows}
\usepackage{array}
%
\usepackage{tkz-euclide}
\usetkzobj{all}
%
%\usepackage{dsfont}
%
\usepackage{xcolor}
\definecolor{space}{HTML}{1F2C4E}
\definecolor{earth}{HTML}{0089FA}
\definecolor{mars}{HTML}{DC7B4E}
\definecolor{dida}{HTML}{FFDE00}
\definecolor{title}{HTML}{FBA706}
%
\usepackage{amsmath}
%
\usepackage{fontspec}
\setmainfont{Open Dyslexic}
%
\title{Radiazione cosmica secondaria}
%
\begin{document}
	\tikzset{partial ellipse/.style args = {#1:#2:#3}{insert path={+ (#1:#3) arc (#1:#2:#3)}},
		line/.style = { draw, ultra thick, ->, shorten >=2pt },	
		particle/.style = { circle, draw=black, ultra thick, fill=dida, text width=8em, text centered, minimum height=5em },
		plus/.style = { circle, draw=black, ultra thick, fill=mars, text width=8em, text centered, minimum height=5em },
		menus/.style = { circle, draw=black, ultra thick, fill=space, text width=8em, text centered, minimum height=5em },
		neutro/.style = { circle, draw=black, ultra thick, fill=white, text width=8em, text centered, minimum height=5em },
		opt/.style = { rectangle, draw=black, ultra thick, fill=earth!50!white, text width=12em, text centered, minimum height=5em },
	}
	\begin{tikzpicture}[background rectangle/.style={fill=white},show background rectangle,>={[inset=0,angle'=27]Stealth},my arrow/.style={decorate,decoration={markings,mark=at position 0.5 with {\arrow[scale=1.5]{>}};}}]
		%title
		\draw [black,ultra thick,fill=title] (0,7) rectangle (30,15);
		\node (example-textwidth-2) [right, align=center, text width=30cm, color=black, font=\fontsize{65pt}{66pt}\selectfont] at (0,11) {Formazione della radiazione cosmica secondaria};
		\begin{scope}[shift={(0,4)}]
			\tkzDefPoint(15,-10){N1}
			\node (inizio) [plus, align=center, font=\fontsize{23pt}{24pt}\selectfont] at (15,0) {p};
			\node (collisioni1) [opt, align=center, font=\fontsize{23pt}{24pt}\selectfont] at (15,-5) {Collisioni nucleari};
			\node (neutrone) [neutro, align=center, font=\fontsize{23pt}{24pt}\selectfont] at (0,-15) {n};
			\tkzDefPoint(0,-22){N6}
			\node (protone1) [plus, align=center, font=\fontsize{23pt}{24pt}\selectfont] at (6,-15) {p};
			\node (pione1) [neutro, align=center, font=\fontsize{23pt}{24pt}\selectfont] at (12,-15) {$\Pi^0$};
			\tkzDefPoint(12,-19){N2}
			\node (pione2) [plus, align=center, font=\fontsize{23pt}{24pt}\selectfont] at (18,-15) {$\Pi^+$};
			\tkzDefPoint(18,-25){N4}
			\node (pione3) [menus, align=center, font=\color{white}\fontsize{23pt}{24pt}\selectfont] at (24,-15) {$\Pi^-$};
			\tkzDefPoint(24,-30){N5}
			\node (antimateria) [opt, align=center, font=\fontsize{18pt}{19pt}\selectfont] at (30,-15) {Antimateria};
			\tkzDefPoint(30,-22){N7}
			\node (collisioni2) [opt, align=center, font=\fontsize{18pt}{19pt}\selectfont] at (6,-28) {Collisioni nucleari};
			\tkzDefPoint(6,-30){N3}
			\node (pione4) [neutro, align=center, font=\fontsize{23pt}{24pt}\selectfont] at (5,-33) {$\Pi^0$};
			\node (pione5) [plus, align=center, font=\fontsize{23pt}{24pt}\selectfont] at (10,-33) {$\Pi^+$};
			\node (pione6) [menus, align=center, font=\color{white}\fontsize{23pt}{24pt}\selectfont] at (15,-33) {$\Pi^-$};
			\node (protone2) [plus, align=center, font=\fontsize{15pt}{16pt}\selectfont] at (0,-33) {p};
			\node (gamma1) [particle, align=center, font=\fontsize{23pt}{24pt}\selectfont] at (8,-22) {$\gamma$};
			\node (gamma2) [particle, align=center, font=\fontsize{23pt}{24pt}\selectfont] at (16,-22) {$\gamma$};
			\node (coppia) [opt, align=center, font=\fontsize{18pt}{19pt}\selectfont] at (12,-26) {Produzione di coppia};
			\node (muone1) [plus, align=center, font=\fontsize{23pt}{24pt}\selectfont] at (16,-28) {$\mu^+$};
			\node (neutrino1) [neutro, align=center, font=\fontsize{23pt}{24pt}\selectfont] at (20,-28) {$\nu$};
			\node (muone2) [menus, align=center, font=\color{white}\fontsize{23pt}{24pt}\selectfont] at (22,-33) {$\mu^-$};
			\node (neutrino2) [neutro, align=center, font=\fontsize{23pt}{24pt}\selectfont] at (26,-33) {$\bar\nu$};
			%
			\begin{scope}[every path/.style=line]
				\path (inizio) -- (collisioni1);
				\path (collisioni1) -- (N1) -- (0,-10) -- (neutrone);
				\path (collisioni1) -- (N1) -- (6,-10) -- (protone1);
				\path (collisioni1) -- (N1) -- (12,-10) -- (pione1);
				\path (collisioni1) -- (N1) -- (18,-10) -- (pione2);
				\path (collisioni1) -- (N1) -- (24,-10) -- (pione3);
				\path (collisioni1) -- (N1) -- (30,-10) -- (antimateria);
				\path (protone1) -- (collisioni2);
				\path (collisioni2) -- (N3) -- (5,-30) -- (pione4);
				\path (collisioni2) -- (N3) -- (10,-30) -- (pione5);
				\path (collisioni2) -- (N3) -- (15,-30) -- (pione6);
				\path (collisioni2) -- (N3) -- (0,-30) -- (protone2);
				\path (pione1) -- (N2) -- (8,-19) -- (gamma1);
				\path (pione1) -- (N2) -- (16,-19) -- (gamma2);
				\path (gamma1) -- (coppia);
				\path (gamma2) -- (coppia);
				\path (pione2) -- (N4) -- (16,-25) -- (muone1);
				\path (pione2) -- (N4) -- (20,-25) -- (neutrino1);
				\path (pione3) -- (N5) -- (22,-30) -- (muone2);
				\path (pione3) -- (N5) -- (26,-30) -- (neutrino2);
				\path [dashed] (neutrone) -- (N6);
				\path [dashed] (antimateria) -- (N7);
			\end{scope}
		\end{scope}
		%
		\begin{scope}[shift={(0,-37)}]
			\draw [fill=dida, ultra thick] (2,5) rectangle (28,-5);
			\node (example-textwidth-2) [right, align=left, text width=25cm, color=black, font=\fontsize{23pt}{24pt}\selectfont] at (2.5,0) {La particella primaria (generalmente un protone) urta con un nucleo d'ossigeno o di azoto dell'alta atmosfera. Da questa collisione vengono generati neutroni, protoni, pioni ($\pi^0$, $\pi^+$, $\pi^-$), antinuetroni, antiprotoni, kaoni e iperoni. I $\pi^0$ decadono elettromagneticamente in due $\gamma$ (fotoni) e questi ultimi possono materializzarsi in coppie $e^+ \, e^-$ (positroni, elettroni). I pioni carichi possono interagire con altri nuclei presenti nell'atmosfera o decadere in leptoni $\mu$ (muoni) e neutrini $\nu$. Gli elettroni irradiano energia sotto forma di raggi $\gamma$ (radiazione di frenamento). Le linee tratteggiate indicano che altre reazioni possono avvenire.};
		\end{scope}
	\end{tikzpicture}
\end{document}